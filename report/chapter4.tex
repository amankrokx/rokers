\chapter{Description Of Tools And Technologies}

\section{HTML}
Hypertext Markup Language (HTML) is the standard markup language for creating web pages and web applications. With Cascading Style Sheets (CSS) and JavaScript it forms a triad of cornerstone technologies for the World Wide Web. Web browsers receive HTML documents from a web server or from local storage and render them into multimedia web pages. HTML describes the structure of a web page semantically and originally included cues for the appearance of the document.
HTML elements are the building blocks of HTML pages. With HTML constructs, images and other objects, such as interactive forms, may be embedded into the rendered page. It provides a means to create structured documents by denoting structural semantics for text such as headings, paragraphs, lists, links, quotes and other items. HTML elements are delineated by tags, written using angle brackets. Tags such as <img /> and <input /> introduce content into the page directly. Others such as <p>...</p> surround and provide information about document text and may include other tags as sub-elements. Browsers do not display the HTML tags, but use them to interpret the content of the page.
HTML can embed programs written in a scripting language such as JavaScript which affect the behavior and content of web pages. Inclusion of CSS defines the look and layout of content. 

\section{Bootstrap - A CSS Framework}
Cascading Style Sheets (CSS) is a style sheet language used for describing the presentation of a document written in a markup language. Although most often used to set the visual style of web pages and user interfaces written in HTML and XHTML, the language can be applied to any XML document, including plain XML, SVG and XUL, and is applicable to rendering in speech, or on other media. Along with HTML and JavaScript, CSS is a cornerstone technology used by most websites to create visually engaging webpages, user interfaces for web applications, and user interfaces for many mobile applications.

CSS is designed primarily to enable the separation of presentation and content, including aspects such as the layout, colors, and fonts. This separation can improve content accessibility, provide more flexibility and control in the specification of presentation characteristics, enable multiple HTML pages to share formatting by specifying the relevant CSS in a separate .css file, and reduce complexity and repetition in the structural content.

Bootstrap is a free and open-source front-end web framework used for  designing websites and web applications. It contains HTML- and CSS-based design templates for typography, forms, buttons, navigation and other interface components, as well as optional JavaScript extensions. Unlike many web frameworks, it concerns itself with front-end development only.

Bootstrap is the second most-starred project on GitHub, with more than 111,600 stars and 51,500 forks.
\section{Javascript}
JavaScript, often abbreviated as JS, is a high-level, interpreted programming language. It is a language which is also characterized as dynamic, weakly typed, prototype-based and multi-paradigm.

Alongside HTML and CSS, JavaScript is one of the three core technologies of the World Wide Web. JavaScript enables interactive web pages and this is an essential part of web applications. The vast majority of websites use it, and all major web browsers have a dedicated JavaScript engine to execute it.

As a multi-paradigm language, JavaScript supports event-driven, functional, and imperative (including object-oriented and prototype-based) programming styles. It has an API for working with text, arrays, dates, regular expressions, and basic manipulation of the DOM, but the language itself does not include any I/O, such as networking, storage, or graphics facilities, relying for these upon the host environment in which it is embedded.

Initially only implemented client-side in web browsers, JavaScript engines are now embedded in many other types of host software, including server-side in web servers and databases, and in non-web programs such as word processors and PDF software, and in runtime environments that make JavaScript available for writing mobile and desktop applications, including desktop widgets.

Although there are strong outward similarities between JavaScript and Java, including language name, syntax, and respective standard libraries, the two languages are distinct and differ greatly in design; JavaScript was influenced by programming languages such as Self and Scheme.

n interpreters are available for many operating systems. CPython, the reference implementation of Python, is open source software and has a community-based development model, as do nearly all of Python's other implementations. Python and CPython are managed by the non-profit Python Software Foundation. 

\section{MySQL}
MySQL is a free and open-source relational database management system (RDBMS) that is widely used in web applications. It is developed, distributed, and supported by Oracle Corporation.
MySQL is known for its speed, reliability, and flexibility. It is used by many large websites and applications, including WordPress, Drupal, and Wikipedia.
MySQL stores data in tables and uses structured query language (SQL) to access and manipulate the data. It can handle very large amounts of data and can be used on many different operating systems, including Windows, Linux, and macOS.
There are many tools and libraries available for interacting with MySQL from various programming languages, including PHP, Python, and Java. This makes it easy to integrate MySQL into web-based applications.


